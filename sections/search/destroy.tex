\subsection{Thuật toán hủy}

\subsubsection*{Phương pháp xóa yêu cầu tệ nhất} 
Phương pháp xóa yêu cầu tệ nhất và xóa ngẫu nhiên được đề xuất bởi S. Ropke và D. Pisinger \cite{ropke2006adaptive}. Cho một yêu cầu $i$ được phục vụ bởi vài xe trong tập nghiệm $s$, chi phí của yêu cầu \textit{cost} được định nghĩa là hàm $cost(i,s)=f(s)-f_{-i}(s)$ với $f_{-i}(s)$ là chi phí của nghiệm mà không có yêu cầu $i$ (yêu cầu được xóa mà không chuyển đến hàng chờ). Chiến thuật ở đây là ta xóa đi những yêu cầu có chi phí cao và cố gắng thêm lại vào các tuyến với chi phí ít hơn.

Tuy nhiên, chúng ta không xóa đi chính xác các yêu cầu có chi phí cao nhất mà thay vào đó chúng ta chọn ngẫu nhiên một yêu cầu có chi phí cao. Điều này được thực hiện để tránh việc xóa các yêu cầu có chi phí cao nhất liên tục và thuật toán bị mắc trong một nghiệm tối ưu địa phương.

\begin{algorithm}
	\label{alg:worst_removal}
	\caption{Xóa yêu cầu tệ nhất}
	\begin{algorithmic}[1]
		\Require $s \in \text{\{nghiệm chấp nhận được\}}, q \in \mathbb{N}, p \in \mathbb{R}_{+}$
		\While {$q > 0$}
		\State Mảng \textit{L} tất cả các yêu cầu \textit{i}, sắp xếp theo thứ tự giảm dần của \textit{cost(i,s)};
		\State chọn một số ngẫu nhiên \textit{y} trong khoảng $[0, 1)$;
		\State yêu cầu $r = L\left[ y^p |L| \right]$;
		\State xóa $r$ khỏi $s$;
		\State $q = q-1$;
		\EndWhile
	\end{algorithmic}
\end{algorithm}

\subsubsection*{Phương pháp xóa ngẫu nhiên}
Thuật toán xóa ngẫu nhiên đơn giản chọn ngẫu nhiên $q$ yêu cầu và loại bỏ chúng khỏi nghiệm hiện tại. Kỹ thuật này có thể coi là một trường hợp đặc biệt của phương pháp xóa Shaw với $p=1$ (S. Ropke, D. Pisinger \cite{ropke2006adaptive}).

\subsection*{Phương pháp xóa yêu cầu ngẫu nhiên với tuyến}
Một hạn chế của thuật toán xóa ngẫu nhiên đó là sau khi chọn ngẫu nhiên yêu cầu cần xóa, ta cần tìm tuyến đường chứa yêu cầu đó để tiến hành phép xóa yêu cầu. Việc tìm tuyến đường khá tốn chi phí khi ta cần duyệt qua toàn bộ các tuyến và duyệt qua từng yêu cầu trong tuyến đó (cỡ $O(n)$). Thay vì chọn ngẫu nhiên yêu cầu trong tập tất cả các yêu cầu, ta chọn ngẫu nhiên một tuyến và sau đó chọn ngẫu nhiên một yêu cầu trong tuyến đó. Tuyến được chọn cũng như vị trí xóa đã được lưu lại vào bộ nhớ khiến cho việc xóa yêu cầu nhanh hơn rất nhiều. Xác xuất chọn không như nhau cho mọi yêu cầu (giống như phương pháp xóa ngẫu nhiên) do ta không tính đến kích thước của tuyến khi chọn ngẫu nhiên. Do đó, cải thiện này không chỉ về mặt kĩ thuật mà còn thay đổi cách thức các yêu cầu được chọn để xóa. Thực nghiệm cho thấy phương pháp xóa này cùng với phương pháp xóa yêu cầu ngẫu nhiên nhiều lần "đóng góp" vào việc tìm ra nghiệm chấp nhận được cũng như nghiệm tốt nhất (đã biết) trong suốt vòng đời của thuật toán.

\subsubsection*{Phương pháp xóa Shaw}
Phương pháp xóa này được phát triển bởi Shaw \cite{shaw1998using}. Cách trình bày trong phần này đã được chỉnh sửa lại để phù hợp với VRPTW. Ý tưởng chung là xóa bỏ các yêu cầu có "liên quan", vì chúng ta hy vọng sẽ dễ dàng thêm lại các yêu cầu tương tự với nhau để tạo ra các nghiệm mới có thể tốt hơn. Nếu chúng ta chọn xóa bỏ các yêu cầu rất khác nhau, thì sau đó, việc thêm các yêu cầu mới sẽ không nhận lại được điều gì do các yêu cầu này có thể chỉ được thêm vào tại vị trí ban đầu của chúng hoặc ở các vị trí tệ. Mức độ liên quan giữa 2 yêu cầu \textit{i} và \textit{j} được định nghĩa dựa trên chỉ số độ liên quan $R(i,j)$. Chỉ số này càng thấp thì 2 yêu cầu càng "giống" nhau.

\begin{algorithm}
	\caption{Xóa yêu cầu - Shaw}
	\begin{algorithmic}[1]
		\Require $s \in \text{\{nghiệm chấp nhận được\}}, q \in \mathbb{N}, p \in \mathbb{R}_{+}$
		\State yêu cầu $r$ được chọn ngẫu nhiên từ $s$;
		\State tập hợp yêu cầu $\mathbb{D}=\{r\}$;
		\While {$|\mathbb{D}| < q$}
		\State $r$ được chọn ngẫu nhiên từ $\mathbb{D}$;
		\State Mảng $L$ chứa tất cả các yêu cầu thuộc $s$ và không thuộc $\mathbb{D}$;
		\State sắp xếp \textit{L} sao cho $i<j \Rightarrow R(r, L\left[ i \right]) < R(r, L\left[ j \right])$;
		\State chọn một số ngẫu nhiên \textit{y} trong khoảng [0, 1);
		\State $\mathbb{D}=\mathbb{D}\cup {L \left[ y^p|L| \right]}$
		\EndWhile
		\State xóa các yêu cầu trong $\mathbb{D}$ khỏi $s$;
	\end{algorithmic}
\end{algorithm}

Chỉ số độ tương đồng được sử dụng bao gồm phụ thuộc vào ba điều kiện là khoảng cách, thời gian, khối lượng. Các điều kiện này được đánh trọng số và ký hiệu lần lượt là $\varphi$, $\chi$ và $\psi$. Chỉ số độ tương đồng được tính như sau
\begin{equation}
	\label{eq:shaw_related}
	R(i,j) = \varphi d_{ij} + \chi |t_{i}-t_{j}| + \psi|l_i - l_j|
\end{equation}
$d_{ij}$ là khoảng cách từ $i$ tới $j$, $t_i$ là thời gian khi đến địa điểm $i$, $l_i$ là tải của xe tại $i$.

Mức độ liên quan được sử dụng để xóa các yêu cầu được mô tả trong Algorithm 3. Ban đầu, một yêu cầu được chọn ngẫu nhiên. Trong các vòng lặp tiếp theo, thuật toán sẽ thực hiện xóa các yêu cầu "giống" với các yêu cầu đã được xóa. Tham số $p \geqslant 1$ biểu diễn cho sự ngẫu nhiên trong cách lựa chọn yêu cầu (\textit{p} càng thấp thì độ ngẫu nhiên càng cao).

Tinh thần của thuật toán Shaw là cố gắng bỏ đi tốp các yêu cầu có độ đo liên quan $R(i,j)$ nhỏ. Tốp các yêu cầu này được kiểm soát bằng tham số $p$. Tuy nhiên, việc sắp xếp các yêu cầu (mảng $L$) theo thứ tự tăng dần của $R(r, L[i])$ là phép toán tốn chi phí. Trong thực tế cài đặt, chúng ta có thể sử dụng cấu trúc dữ liệu \textit{min heap} để lấy ra tốp các yêu cầu có độ đo liên quan nhỏ nhất.

\subsubsection*{Hủy tuyến tệ nhất}
Thuật toán hủy tuyến tệ nhất được tác giả đề xuất trong luận văn này. Tinh thần chung của thuật toán là chúng ta cố gắng bỏ đi toàn bộ yêu cầu trên một hoặc một vài tuyến đường nào đó. Việc bỏ đi cả tuyến cũng tương đương với việc làm giảm số xe cần để phục vụ khách hàng. Mục tiêu của CVRPTW là giảm tổng khoảng cách di chuyển và nghiệm thu được phải thỏa mãn các ràng buộc về tải trọng, số xe cũng như khung thời gian. Tuy nhiên, trong thực tế vận hành của một doanh nghiệp, giảm bớt được số xe phục vụ là điều rất có ý nghĩa bởi chi phí mua (thuê) xe là đắt đỏ so với việc di chuyển xa hơn một vài phần trăm.

Áp dụng tư tưởng \textit{tham lam}, cho rằng các tuyến nên được bỏ đi là các tuyến có chi phí trung bình trên mỗi khách hàng cao. Chúng ta kì vọng có thể thêm lại các khách hàng đó vào các tuyến khác "thoáng" hơn hay là chi phí trung bình trên mỗi khách hàng thấp hơn. Chi phí trung bình trên mỗi yêu cầu được cho bởi công thức
\begin{equation}
	\label{eq:destroy_route}
	\text{avg\_cost}(r, s) = \frac{\text{cost}(r, s)}{\text{size}(r)-1}
\end{equation}
với $r$ là tuyến được chọn, $\text{cost}(r,s)$ là chi phí của tuyến $r$ trong nghiệm $s$, $\text{size}(r)$ là số phần tử của tuyến $r$. Lưu ý rằng tuyến bắt đầu và kết thúc tại nút $0$.

Ngoài ra để tránh việc bỏ đi tuyến "tệ" một cách cứng nhắc, tác giả đưa vào một nhiễu nhỏ. Khi đó, công thức tính chi phí trung bình trên mỗi yêu cầu được thay đổi thành
\begin{equation}
	\label{eq:destroy_route}
	\text{avg\_cost}(r, s) = \frac{\text{cost}(r, s)}{\text{size}(r)-1} + \lambda \phi d_{max}3w
\end{equation}
với $d_{max}$ là khoảng cách lớn nhất giữa hai yêu cầu, $\phi$ là một số ngẫu nhiên trong khoảng $(-1,1)$ và $\lambda$ là một hằng số điều khiển.

Ngoài ra, chúng ta cũng không muốn bỏ đi các tuyến đang phục vụ nhiều yêu cầu vì khi thêm lại thuật toán sẽ mất rất nhiều thời gian. Chính vì thế, tác giả chỉ bỏ đi những tuyến có số yêu cầu ít hơn số yêu cầu trung bình trên mỗi tuyến của nghiệm hiện tại.

Cuối cùng, ta bỏ đi toàn bộ yêu cầu trên tuyến có số yêu cầu ít hơn số yêu cầu trung bình trên tất cả các tuyến và có chi phí trung bình trên mỗi yêu cầu cao nhất.