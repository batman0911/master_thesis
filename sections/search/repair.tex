\subsection{Thuật toán sửa}


\subsection{Phương pháp tham lam cơ bản}
\label{sec:basic_greedy}
Heuristic tham lam cơ bản (\textit{basic greedy}) là 1 kỹ thuật xây dựng đơn giản. Nó thực hiện tối đa $n$ lần lặp, chèn thêm 1 yêu cầu trong mỗi bước. Với $\Delta f_{i, k}$ biểu diễn cho sự thay đổi hàm mục tiêu bằng cách chèn thêm yêu cầu $i$ vào tuyến đường $k$ tại vị trí mà giá trị tăng thêm của hàm mục tiêu là nhỏ nhất. Nếu không chèn yêu cầu $i$ vào tuyến đường $k$ thì ta đặt $\Delta f_{i, k} = \infty$ và $c_i = \min_{k \in K}\{\Delta f_{i, k}\}$. Nói cách khác, $c_i$ là chi phí khi chèn thêm yêu cầu $i$ vào vị trí tốt nhất của nghiệm hiện tại, gọi là vị trí chi phí nhỏ nhất. Cuối cùng, ta chọn yêu cầu $i$ với $\min_{i \in U} c_i$ và chèn nó vào vị trí chi phí nhỏ nhất. Quá trình này tiếp tục cho đến khi tất cả các yêu cầu được thêm hoặc không còn yêu cầu nào nữa.

Trong mỗi vòng lặp, thuật toán chỉ thay đổi một tuyến đường (tuyến mà yêu cầu mới được thêm vào), và ta không cần phải tính toán lại chi phí chèn thêm trong tất cả các tuyến đường khác. Điểm này giúp tăng hiệu năng cho thuật toán. Đặc biệt nếu hàm mục tiêu là tổng quãng đường đi chuyển thì chúng ta có thể tính toán rất nhanh chi phí tăng thêm này. Giả sửa ta cần chèn thêm yêu cầu $u$ và giữa hai yêu cầu $i$ và $j$ trong tuyến đường $k$. Khi đó, chi phí tăng thêm của việc chèn thêm yêu cầu $u$ vào giữa $i$ và $j$ là $\Delta f_{u, i, j, k} = d_{iu} + d_{uj} - d_{ij}$.

Dễ dàng nhận thấy vấn đề với cách tiếp cận này là nó thường trì hoãn việc đặt các yêu cầu có chi phí cao cho các lần lặp cuối cùng, nơi chúng ta không có nhiều cơ hội cho việc chèn thêm yêu cầu vì nhiều tuyến đường đều đã kín. 

\subsection{Phương pháp tham lam với nhiễu ngẫu nhiên}

Để giải quyết một phần hạn chế của phương pháp tham lam cơ bản, ta có thể thêm một chút nhiễu vào hàm mục tiêu. Điều này giúp thuật toán có thể không chọn chèn yêu cầu vào tuyến làm tăng chi phí ít nhất mà có thể là ít thứ hai hoặc thứ ba. 
\begin{equation}
  \text{noise} = \lambda p d_{\text{max}}
\end{equation}
với $d_{\text{max}}$ là khoảng cách lớn nhất giữa hai yêu cầu, $p$ là một số ngẫu nhiên trong khoảng $(-1,1)$ và $\lambda$ là một hằng số điều khiển. Sau đó ta gán
\begin{equation}
  \Delta f_{u, i, j, k} := \Delta f_{u, i, j, k} + \text{noise}
\end{equation}

\subsection{Phương pháp regret heuristic}
Regret heuristic đã được sử dụng bởi Potvin và Rousseau (1993) \cite{} cho VRPTW. Phương pháp này cố gắng cải thiện nhược điểm của kỹ thuật tham lam bằng cách kiểm tra lại kết quả sau khi chọn chèn thêm yêu cầu. Đặt $x_{ik} \in \{1, ..., m\}$ là biến biểu diễn tuyến đường cho yêu cầu $i$ có chi phí chèn thêm vào thấp thứ $k$, điều này có nghĩa là $\Delta f_{i, x_{ik}} \leqslant \Delta f_{i, x_{ik'}}$. Sử dụng ký hiệu này, ta có thể biểu diễn $c_i$ từ phần \ref{sec:basic_greedy}, $c_i = \Delta f_{i, x_{i1}}$. Trong phương pháp này, chúng ta có thể định nghĩa 1 giá trị \textit{regret} $c_i^* = \Delta f_{i, x_{i2}} - \Delta f_{i, x_{i1}}$ . Nói cách khác, giá trị regret là khoảng cách giữa chi phí của việc chèn thêm yêu cầu vào tuyến đường tốt nhất so với tuyến đường tốt thứ 2. Trong mỗi vòng lặp, thuật toán chọn ra yêu cầu $i$ thỏa mãn điều kiện $max_{i \in U} \{c_i^*\}$. Với điều kiện này, yêu cầu được đảm bảo thêm vào vị trí có chi phí nhỏ nhất. Điều này khiến cho các ràng buộc bị phá vỡ. Nói cách khác, thuật toán sẽ thực hiện chèn yêu cầu vào phương án nếu không sẽ hối tiếc khi không thực hiện điều này.

Phương pháp này có thể mở rộng 1 cách tự nhiên để định nghĩa 1 lớp các phương pháp regret heuristic gọi là phương pháp k-regret heuristic là 1 phương pháp mà mỗi lần thêm yêu cầu vào phương án cần phải thỏa mãn điều kiện
\begin{equation}
    \max\limits_{i \in U} \{ \sum_{j=1}^k (\Delta f_{i, x_{ij}} - \Delta f_{i, x_{i1}}) \}
\end{equation}
Nếu 1 vài yêu cầu không thể được chèn thêm vào ít nhất $m-k+1$ tuyến đường thì yêu cầu đó sẽ được chèn vào số lượng tuyến đường ít nhất (có thể là 1). Các ràng buộc bị phá vỡ bởi việc lựa chọn yêu cầu với chi phí chèn tốt nhất. Yêu cầu được chèn vào vị trí ít chi phí nhất. Với $k>2$ thì thuật toán sẽ tính toán chi phí của việc thêm vào 1 yêu cầu qua k tuyến đường tốt nhất và chèn yêu cầu mà khoảng cách chi phí giữa việc thêm nó vào tuyến đường tốt nhất với tuyến đường tốt thứ $k-1$ là lớn nhất. So sánh với 2-regret heuristic, thuật toán với giá trị lớn của $k$ khám phá ra sớm hơn khả năng bị giới hạn khi thêm 1 yêu cầu.

Với việc xét top-k vị trí tốt nhất của mỗi yêu cầu, hiệu năng của \textit{regret-k} sẽ không tốt bằng \textit{basic greedy}. Để dễ hình dung, với $n$ vị trí có thể chèn yêu cầu, trong mỗi bước lặp của thuật toán, với \textit{basic greedy} ta cần duyệt $O(n)$ lần qua các vị trí để tìm ra vị trí chèn tốt nhất; với \textit{regret-k} nếu sử dụng heap ta mất $O(k \times n)$ lần duyệt để  lấy ra top-k và $O(k)$ lần duyệt nữa để lấy ra yêu cầu mà khoảng cách $\Delta f$ trong top-k lớn nhất. Trong thực tế cài đặt, tác giả sử dụng đến $k=5$ và regret-k chứng tỏ được nó là một phương pháp mạnh và đáng để đánh đổi một chút về mặt hiệu năng. 