\section{Tìm kiếm lân cận rộng thích ứng}
Tìm kiếm lân cận rộng thích ứng (\textit{adaptive large neighbourhood search - ALNS}) được giới thiệu bởi Ropke, Pisinger (2006) \cite{ropke2006adaptive} là một mở rộng của LNS. Thay vì chỉ sử dụng một chiến thuật hủy và một chiến thuật thêm lại yêu cầu như LNS, ALNS cho phép lựa chọn nhiều toán tử hủy và thêm lại. Việc này cho phép thuật toán tìm kiếm không gian nghiệm một cách linh hoạt hơn và khó bị bẫy ở một nghiệm tối ưu cục bộ. Điểm thú vị của ALNS là các thuật toán hủy và thêm lại không được chọn một cách ngẫu nhiên mà lựa chọn có trọng số phụ thuộc vào trạng thái (nghiệm hiện tại) của bài toán.

\subsection{Lựa chọn phương pháp xóa và thêm lại}
Để lựa chọn phương pháp xóa và thêm lại, ta gán cho mỗi heuristic một trọng số khác nhau và sử dụng nguyên tắc "bánh xe lựa chọn". Nếu có $k$ heuristic với trọng số $w_i, i \in \{1,...,k\}$, ta chọn heuristic $j$ với xác suất:
\begin{equation}
  \label{eq:select}
  p_j = \frac{w_j}{\sum_{i=1}^k w_i}
\end{equation}
Lưu ý rằng việc lựa chọn thuật toán xóa và thêm lại là độc lập với nhau. Các trọng số này có thể được thiết lập thủ công và không đổi trong suốt vòng đời của việc tìm kiếm hoặc nó có thể được điều chỉnh tự động để "thích ứng" với trạng thái hiện tại của hệ. Một cách điều chỉnh các tham số này tự động được trình bày ngay sau đây.

\subsection{Điều chỉnh tham số tự động}
 Trọng số được điều chỉnh mỗi khi có nghiệm mới được chấp nhận. Ý tưởng chung là theo dõi một điểm số đại diện cho độ hiệu quả của thuật toán trong các vọng lặp gần đây. Điểm số càng cao thì thuật toán được chọn càng hiệu quả. Quá trình tìm kiếm được chia thành nhiều bước, mỗi bước là một số vòng lặp. Điểm của mỗi heuristic được đặt là $0$ khi bắt đầu và được tăng thêm $\sigma_1, \sigma_2, \sigma_3$ tùy thuộc vào tình huống
\begin{table}[caption={Tham số cập nhật trọng số}, label=tab:weight]
  \begin{tabularx}{\textwidth}{|l|X|}
    \hline
    Tham số & Mô tả \\ \hline
    $\sigma_1$ & Hành động xóa-chèn cuối cùng dẫn đến một nghiệm mới tốt hơn nghiệm tốt nhất toàn cục \\ \hline 
    $\sigma_2$ & Hành động xóa-chèn cuối cùng dẫn đến một nghiệm chưa được chấp nhận trước đó, chi phí tốt hơn chi phí của nghiệm hiện tại \\ \hline
    $\sigma_3$ & Hành động xóa-chèn cuối cùng dẫn đến một nghiệm chưa được chấp nhận trước đó, chi phí của nghiệm mới tệ hơn chi phí của nghiệm hiện tại nhưng thỏa mãn điều kiện chấp nhận nghiệm \\ \hline
    \end{tabularx}
\end{table}
Trong mỗi bước, thao tác xóa và thêm lại được cập nhật một lượng như nhau vì ta không chắc sự "cải thiện" nghiệm đến từ việc xóa hay thêm lại yêu cầu. Mỗi khi kết thúc bước, ta tính toán lại trọng số mới sử dụng các điểm sô trên. Gọi $\omega_{ij}$ là trọng số của heuristic $i$ trong bước $j$. Ta đánh các trọng số như nhau tại bước đầu tiên, sau đó khi bước $j$ kết thúc, ta tính lại trọng số cho tất cả các heuristic để sử dụng cho bước $j+1$ như sau:
\begin{equation}
  \label{eq:adaptive_weight}
  \omega_{i, j+1} = \omega_{ij}(1-r)+r\frac{\pi_i}{\theta_i}
\end{equation}
Trong đó $\pi_i$ là điểm số của heuristic $i$ trong bước $j$ và $\theta_i$ là số lần ta cố gắng sử dụng heuristic $i$ trong bước $j$. Tham số $r$ là tham số điều khiển tốc độ điều chỉnh trọng số. Nếu $r=0$, nghĩa là chúng ta không sử dụng điểm để điều chỉnh trọng số hay nói cách khác là các trọng số được giữ nguyên từ trong suốt quá trình tìm kiếm. Nếu $r=1$, nghĩa là ta lấy điểm thu được từ bước gần nhất để quyết định trọng số. 

Việc điều chỉnh trọng số như trên làm tăng xác suất chọn thuật toán xóa (chèn) đã mang lại hiệu quả ở bước trước đó. Về cơ bản, với việc sử dụng chiến thuật điều chỉnh trọng số như trên, ta kì vọng rằng các thuật toán xóa (chèn) đã hiệu quả ở bước trước thì cũng sẽ hiệu quả ở bước tiếp theo. 

\subsection{Thêm nhiễu khi chỉnh tham số tự động}
Như đã trình bày, với việc sử dụng chiến thuật lựa chọn tham số tự động như trên, ta kì vọng rằng thuật toán nếu đang hiệu quả thì nó sẽ tiếp tục hiệu quả. Tuy nhiên việc cộng một lượng cố định vào thuật toán đó về lâu dài (khi trải qua nhiều vòng lặp) thì trọng số của nó trở lên lớn dẫn đến xác suất lựa chọn thuật toán này cũng lớn theo. Chúng ta cũng chưa sử dụng yếu tố vòng lặp (hay thời gian chạy). Ý tưởng đơn giản là nếu rất "lâu" rồi ta mới có một thuật toán hiệu quả thì ta cũng nên điều chỉnh trọng số của nó theo "thời gian chờ đó". Giả sử sau $m$ vòng lặp, chúng ta mới lại có một nghiệm được chấp nhận từ lần cuối cùng nghiệm được chấp nhận, trọng số của thuật toán sẽ được điều chỉnh một lượng tỉ lệ với $1 - e^{-\gamma m}$. Hàm $\text{exp}$ được sử dụng để chuẩn hóa lượng này trong khoảng $(0,1)$ khi $m$ lớn hoặc nhỏ. Cuối cùng ta có biểu thức cho trọng số của thuật toán như sau:
\begin{equation}
  \label{eq:boost_adaptive_weight}
  \omega_{i, j+1} = \omega_{ij}(1-r)+r\frac{\pi_i} {\theta_i} + \alpha \beta (1 - e^{-\gamma m})
\end{equation}
với $\alpha$ (có thể âm hoặc dương) và $\gamma$ (dương) là các tham số điều khiển, $\beta$ là một số ngẫu nhiên trong khoảng $(0,1)$. 
