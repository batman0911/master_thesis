\section{Chất lượng nghiệm}

\subsection{Dữ liệu nhỏ (Solomon)}

Trước tiên, chúng ta bắt đầu với tập dữ liệu Solomon (1987). Với các cấu hình loại C, ALNS cho nghiệm cách nghiệm tốt nhất đã biết trung bình $0.20\%$ và $0.42\%$ cho C1 và C2. 

\begin{table}[caption={Kết quả đo với tập Solomon C \\
  \scriptsize \textit{ins: cấu hình, cost: chi phí thu được với ALNS, nv: số xe được sử dụng, bkcost: chi phí tốt nhất đã biết, bknv: số xe tốt nhất đã biết, gap (\%): khoảng cách so với nghiệm tốt nhất đã biết}}, label=exp:solomonC]
  \begin{adjustbox}{width=1\textwidth}
  \small
  \begin{tabularx}{\textwidth}{lrrrlllrrrll}
  \hline
  \text{ins} & \multicolumn{1}{l}{\text{cost}} & \multicolumn{1}{l}{\text{nv}} & \multicolumn{1}{l}{\text{bkcost}} & \text{bknv} & \text{gap} & \text{ins} & \multicolumn{1}{l}{\text{cost}} & \multicolumn{1}{l}{\text{nv}} & \multicolumn{1}{l}{\text{bkcost}} & \text{bknv} & \text{gap} \\ \hline
  c101 & 828.94 & 10 & 827.30 & 10 & 0.20 & c201 & 591.56 & 3 & 589.10 & 3 & 0.42 \\ \hline
  c102 & 828.94 & 10 & 827.30 & 10 & 0.20 & c202 & 591.56 & 3 & 589.10 & 3 & 0.42 \\ \hline
  c103 & 828.06 & 10 & 826.30 & 10 & 0.21 & c203 & 591.17 & 3 & 588.70 & 3 & 0.42 \\ \hline
  c104 & 824.78 & 10 & 822.90 & 10 & 0.23 & c204 & 590.60 & 3 & 588.10 & 3 & 0.42 \\ \hline
  c105 & 828.94 & 10 & 827.30 & 10 & 0.20 & c205 & 588.88 & 3 & 586.40 & 3 & 0.42 \\ \hline
  c106 & 828.94 & 10 & 827.30 & 10 & 0.20 & c206 & 588.49 & 3 & 586.00 & 3 & 0.43 \\ \hline
  c107 & 828.94 & 10 & 827.30 & 10 & 0.20 & c207 & 588.29 & 3 & 585.80 & 3 & 0.42 \\ \hline
  c108 & 828.94 & 10 & 827.30 & 10 & 0.20 & c208 & 588.32 & 3 & 585.80 & 3 & 0.43 \\ \hline
  c109 & 828.94 & 10 & 827.30 & 10 & 0.20 &  &  &  &  &  &  \\ \hline
  avg & & & & & 0.20 &  &  &  &  & & 0.42 \\ \hline
  \end{tabularx}
  \end{adjustbox}
  \end{table}
  Thí nghiệm được thiết lập có thời gian timeout 1 phút, chạy 5 lần và lấy kết quả tốt nhất. Tập C1 và C2 tương đối nhỏ và đã phân cụm nên trong thực tế thuật toán chạy rất nhanh để ra được nghiệm tối ưu và không có sự khác biệt giữa các lần chạy, trên CPU được thí nghiệm, ALNS mất dưới 1 giây để tìm ra nghiệm tối ưu.


  \begin{table}[caption={Kết quả đo với tập Solomon R1}, label=exp:solomonR1]
    % \begin{adjustbox}{width=1\textwidth}
    \centering
    \begin{tabular}{lrrrll}
    \hline
    instance & \multicolumn{1}{l}{alns best} & \multicolumn{1}{l}{nv} & \multicolumn{1}{l}{bk cost} & bk nv & gap (\%) \\ \hline
    \text{r101} & 1,642.88 & 20 & \text{1,637.70} & 20 & \text{0.32} \\ \hline
    \text{r102} & 1,472.81 & 18 & \text{1,466.60} & 18 & \text{0.42} \\ \hline
    \text{r103} & 1,213.62 & 15 & \text{1,208.70} & 14 & \text{0.41} \\ \hline
    \text{r104} & 976.61 & 11 & \text{971.50} & 11 & \text{0.53} \\ \hline
    \text{r105} & 1,360.78 & 15 & \text{1,355.30} & 15 & \text{0.40} \\ \hline
    \text{r106} & 1,239.37 & 13 & \text{1,234.60} & 13 & \text{0.39} \\ \hline
    \text{r107} & 1,073.60 & \text{12} & \text{1,064.60} & \text{11} & \text{0.85} \\ \hline
    \text{r108} & 944.44 & 10 & \text{932.10} & 10 & \text{1.32} \\ \hline
    \text{r109} & 1,152.38 & 13 & \text{1,146.90} & 13 & \text{0.48} \\ \hline
    \text{r110} & 1,078.59 & 12 & \text{1,068.00} & 12 & \text{0.99} \\ \hline
    \text{r111} & 1,053.50 & 12 & \text{1,048.70} & 12 & \text{0.46} \\ \hline
    \text{r112} & \text{955.68} & 10 & \text{948.60} & \text{10} & \text{0.75} \\ \hline
    avg &  &  &  &  & 0.61 \\ \hline
    \end{tabular}
    % \end{adjustbox}
  \end{table}


  \begin{table}[caption={Kết quả đo với tập Solomon R2}, label=exp:solomonR2]
    % \begin{adjustbox}{width=1\textwidth}
    \centering
    \begin{tabular}{lrrrll}
    \hline
    instance & \multicolumn{1}{l}{alns best} & \multicolumn{1}{l}{nv} & \multicolumn{1}{l}{bk cost} & bk nv & gap (\%) \\ \hline
    r201 & 1,152.96 & \textbf{7} & 1,143.20 & 8 & 0.32 \\ \hline
    r202 & 1,035.32 & \textbf{7} & 1,029.60 & 8 & 0.42 \\ \hline
    r203 & 880.90 & 6 & 870.80 & 6 & 0.41 \\ \hline
    r204 & 743.91 & \textbf{4} & 731.30 & 5 & 0.53 \\ \hline
    r205 & 958.81 & 5 & 949.80 & 5 & 0.40 \\ \hline
    r206 & 883.92 & 5 & 875.90 & 5 & 0.39 \\ \hline
    r207 & 806.31 & 5 & 794.00 & 4 & 0.85 \\ \hline
    r208 & 948.57 & 4 & 701.00 & 4 & 1.77 \\ \hline
    r209 & 717.53 & 5 & 854.80 & 5 & 0.48 \\ \hline
    r210 & 909.32 & \textbf{5} & 900.50 & 6 & 0.99 \\ \hline
    r211 & 1,053.50 & 5 & 746.70 & 4 & 0.46 \\ \hline
    avg &  &  &  &  & 1.38 \\ \hline
    \end{tabular}
    % \end{adjustbox}
  \end{table}

  Tập R1 và R2 có các yêu cầu được tạo hoàn toàn ngẫu nhiên thế nên cũng có nhiều nghiệm chấp nhận được và thuật toán cũng khó bị bẫy ở một nghiệm tối ưu cục bộ. Tuy nhiên, thuật toán cũng mất nhiều thời gian để tìm nghiệm tối ưu hơn do có quá nhiều nghiệm thỏa mãn các ràng buộc. Riêng với tập R2, ALNS đã tìm ra nghiệm với số xe ít hơn nghiệm tốt nhất đã biết mà tổng khoảng cách chỉ chênh lệch nhỏ. 

  \begin{table}[caption={Kết quả đo với tập Solomon RC1}, label=exp:solomonRC1]
    % \begin{adjustbox}{width=1\textwidth}
    \centering
    \begin{tabular}{lrrrll}
    \hline
    instance & \multicolumn{1}{l}{alns best} & \multicolumn{1}{l}{nv} & \multicolumn{1}{l}{bk cost} & bk nv & gap (\%) \\ \hline
    rc101 & 1,623.58 & 16 & 1,619.80 & 15 & 0.23 \\ \hline
    rc102 & 1,461.23 & 14 & 1,457.40 & 14 & 0.26 \\ \hline
    rc103 & 1,266.62 & 11 & 1,258.00 & 11 & 0.69 \\ \hline
    rc104 & 1,136.91 & 10 & 1,132.30 & 10 & 0.41 \\ \hline
    rc105 & 1,518.58 & 16 & 1,513.70 & 15 & 0.32 \\ \hline
    rc106 & 1,376.99 & 13 & 1,372.70 & 12 & 0.31 \\ \hline
    rc107 & 1,211.11 & 12 & 1,207.80 & 12 & 0.27 \\ \hline
    rc108 & 1,118.13 & 11 & 1,114.20 & 11 & 0.35 \\ \hline
    avg &  &  &  &  & 0.36 \\ \hline
    \end{tabular}
    % \end{adjustbox}
  \end{table}

  \begin{table}[caption={Kết quả đo với tập Solomon RC2}, label=exp:solomonRC2]
    % \begin{adjustbox}{width=1\textwidth}
    \centering
    \begin{tabular}{lrrrll}
    \hline
    instance & \multicolumn{1}{l}{alns best} & \multicolumn{1}{l}{nv} & \multicolumn{1}{l}{bk cost} & bk nv & gap (\%) \\ \hline
    rc201 & 1,274.61 & \textbf{8} & 1,261.80 & 9 & 1.02 \\ \hline
    rc202 & 1,099.54 & \textbf{6} & 1,092.30 & 8 & 0.66 \\ \hline
    rc203 & 931.16 & 5 & 923.70 & 5 & 0.81 \\ \hline
    rc204 & 788.66 & 4 & 783.50 & 4 & 0.66 \\ \hline
    rc205 & 1,157.66 & 7 & 1,154.00 & 7 & 0.32 \\ \hline
    rc206 & 1,060.50 & \textbf{6} & 1,051.10 & 7 & 0.89 \\ \hline
    rc207 & 966.08 & 6 & 962.90 & 6 & 0.33 \\ \hline
    rc208 & 785.73 & 4 & 776.10 & 4 & 1.24 \\ \hline
    avg &  &  &  &  & 0.74 \\ \hline
    \end{tabular}
    % \end{adjustbox}
  \end{table}

  Tập RC1 và RC2 chứa các yêu cầu ở các vị trí được phân cụm một cách tương đối (lai giữa tập R và tập C) được đánh giá là tập khó, do các vị trí của yêu cầu đủ ngẫu nhiên để bài toán có nhiều nghiệm chấp nhận được nhưng lại dễ bị bẫy ở nghiệm tối ưu cục bộ khi mà xe đã phụ vụ các yêu cầu ở trong một cụm chỉ cần ta bỏ đi một (vài) yêu cầu "quan trọng" trong cụm là đã nhận được một nghiệm tệ hơn trước khá nhiều. Mặc dù vậy ALNS vẫn tỏ ra hiệu quả khi tìm được nghiệm với chênh lệch trung bình $0.36\%$ và $0.74\%$ cho RC1 và RC2. Ngoài ra, ALNS cũng tìm được nghiệm với số xe ít hơn nghiệm tốt nhất đã biết trong một số cấu hình của tập RC2.