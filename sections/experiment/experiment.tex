\chapter{Thực nghiệm và kết quả}

Trong phần này, chúng ta sẽ xem xét kết quả thực nghiệm thu được khi áp dụng ALNS cho VRPTW với hai tập dữ liệu của Solomon (1987) và Homberger \& Gehring (1999). Ở cả hai tập, các bộ dữ liệu được chia thành 3 loại C, R và RC. Với dữ liệu lớp C, các yêu cầu được phân thành các cụm rõ rệt, lớp R là hoàn toàn ngẫu nhiên và lớp RC là sự kết hợp của hai lớp trên. Tác giả chỉ xem xét các tập từ 100 yêu cầu trở lên (bỏ qua tập Solomon 25, 50 yêu cầu vì nhìn chung số lượng yêu cầu như vậy là quá nhỏ để nhận thấy sự khác biệt khi so sánh ALNS với các thuật toán khác). 

Tất cả các thí nghiệm được chạy trên CPU Intel(R) Xeon(R) CPU E5-2680 v4 @ 2.40GHz với Ubuntu 22.04.1 LTS. Mã nguồn được viết bằng ngôn ngữ C++ và được biên dịch bằng GCC 11.4.0 với các tùy chọn tối ưu hóa -O3.