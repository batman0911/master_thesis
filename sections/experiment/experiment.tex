\chapter{Thực nghiệm và kết quả}

Trong phần này, chúng ta sẽ xem xét kết quả thực nghiệm thu được khi áp dụng ALNS cho VRPTW với hai tập dữ liệu của Solomon (1987) và Homberger \& Gehring (1999). Ở cả hai tập, các bộ dữ liệu được chia thành 3 loại C, R và RC. Với dữ liệu lớp C, các yêu cầu được phân thành các cụm rõ rệt, lớp R là hoàn toàn ngẫu nhiên và lớp RC là sự kết hợp của hai lớp trên. Tác giả chỉ xem xét các tập từ 100 yêu cầu trở lên (bỏ qua tập Solomon 25, 50 yêu cầu vì nhìn chung số lượng yêu cầu như vậy là quá nhỏ để nhận thấy sự khác biệt khi so sánh ALNS với các thuật toán khác).

\begin{figure}[H] % places figure environment here   
	\label{fig:perf_ct_c1}
	\begin{subfigure}{.3\textwidth}
		\centering
		\includegraphics[width=1\linewidth]{figures/cls_c.png}
		\caption{C-class}
		\label{fig:cls_c}
	\end{subfigure}%
	\begin{subfigure}{.3\textwidth}
		\centering
		\includegraphics[width=1\linewidth]{figures/cls_r.png}
		\caption{R-class}
		\label{fig:cls_r}
	\end{subfigure}
	\begin{subfigure}{.3\textwidth}
		\centering
		\includegraphics[width=1\linewidth]{figures/cls_rc.png}
		\caption{RC-class}
		\label{fig:cls_rc}
	\end{subfigure}
	\caption{Lớp các cấu hình}
\end{figure}

\begin{figure}[H] % places figure environment here   
	\label{fig:perf_ct_c1}
	\begin{subfigure}{.3\textwidth}
		\centering
		\includegraphics[width=1\linewidth]{figures/routes_c101.png}
		\caption{C-class}
		\label{fig:route_c}
	\end{subfigure}%
	\begin{subfigure}{.3\textwidth}
		\centering
		\includegraphics[width=1\linewidth]{figures/routes_r101.png}
		\caption{R-class}
		\label{fig:route_r}
	\end{subfigure}
	\begin{subfigure}{.3\textwidth}
		\centering
		\includegraphics[width=1\linewidth]{figures/routes_rc101.png}
		\caption{RC-class}
		\label{fig:route_rc}
	\end{subfigure}
	\caption{Minh họa lời giải cho các lớp cấu hình}
\end{figure}

Tất cả các thí nghiệm được chạy trên CPU Intel(R) Xeon(R) CPU E5-2680 v4 @ 2.40GHz với Ubuntu 22.04.1 LTS. Mã nguồn được viết bằng ngôn ngữ C++ và được biên dịch bằng GCC 11.4.0 với các tùy chọn tối ưu hóa -O3.

\section{Chất lượng nghiệm}

\subsection{Dữ liệu nhỏ (Solomon)}

Trước tiên, chúng ta bắt đầu với tập dữ liệu Solomon (1987). Với các cấu hình loại C, ALNS cho nghiệm cách nghiệm tốt nhất đã biết trung bình $0.20\%$ và $0.42\%$ cho C1 và C2. 

\begin{table}[caption={Kết quả đo với tập Solomon C \\
  \scriptsize \textit{ins: cấu hình, cost: chi phí thu được với ALNS, nv: số xe được sử dụng, bkcost: chi phí tốt nhất đã biết, bknv: số xe tốt nhất đã biết, gap (\%): khoảng cách so với nghiệm tốt nhất đã biết}}, label=exp:solomonC]
  \begin{adjustbox}{width=1\textwidth}
  \small
  \begin{tabularx}{\textwidth}{lrrrlllrrrll}
  \hline
  \text{ins} & \multicolumn{1}{l}{\text{cost}} & \multicolumn{1}{l}{\text{nv}} & \multicolumn{1}{l}{\text{bkcost}} & \text{bknv} & \text{gap} & \text{ins} & \multicolumn{1}{l}{\text{cost}} & \multicolumn{1}{l}{\text{nv}} & \multicolumn{1}{l}{\text{bkcost}} & \text{bknv} & \text{gap} \\ \hline
  c101 & 828.94 & 10 & 827.30 & 10 & 0.20 & c201 & 591.56 & 3 & 589.10 & 3 & 0.42 \\ \hline
  c102 & 828.94 & 10 & 827.30 & 10 & 0.20 & c202 & 591.56 & 3 & 589.10 & 3 & 0.42 \\ \hline
  c103 & 828.06 & 10 & 826.30 & 10 & 0.21 & c203 & 591.17 & 3 & 588.70 & 3 & 0.42 \\ \hline
  c104 & 824.78 & 10 & 822.90 & 10 & 0.23 & c204 & 590.60 & 3 & 588.10 & 3 & 0.42 \\ \hline
  c105 & 828.94 & 10 & 827.30 & 10 & 0.20 & c205 & 588.88 & 3 & 586.40 & 3 & 0.42 \\ \hline
  c106 & 828.94 & 10 & 827.30 & 10 & 0.20 & c206 & 588.49 & 3 & 586.00 & 3 & 0.43 \\ \hline
  c107 & 828.94 & 10 & 827.30 & 10 & 0.20 & c207 & 588.29 & 3 & 585.80 & 3 & 0.42 \\ \hline
  c108 & 828.94 & 10 & 827.30 & 10 & 0.20 & c208 & 588.32 & 3 & 585.80 & 3 & 0.43 \\ \hline
  c109 & 828.94 & 10 & 827.30 & 10 & 0.20 &  &  &  &  &  &  \\ \hline
  avg & & & & & 0.20 &  &  &  &  & & 0.42 \\ \hline
  \end{tabularx}
  \end{adjustbox}
  \end{table}
  Thí nghiệm được thiết lập có thời gian timeout 1 phút, chạy 5 lần và lấy kết quả tốt nhất. Tập C1 và C2 tương đối nhỏ và đã phân cụm nên trong thực tế thuật toán chạy rất nhanh để ra được nghiệm tối ưu và không có sự khác biệt giữa các lần chạy, trên CPU được thí nghiệm, ALNS mất dưới 1 giây để tìm ra nghiệm tối ưu.


  \begin{table}[caption={Kết quả đo với tập Solomon R1}, label=exp:solomonR1]
    % \begin{adjustbox}{width=1\textwidth}
    \centering
    \begin{tabular}{lrrrll}
    \hline
    instance & \multicolumn{1}{l}{alns best} & \multicolumn{1}{l}{nv} & \multicolumn{1}{l}{bk cost} & bk nv & gap (\%) \\ \hline
    \text{r101} & 1,642.88 & 20 & \text{1,637.70} & 20 & \text{0.32} \\ \hline
    \text{r102} & 1,472.81 & 18 & \text{1,466.60} & 18 & \text{0.42} \\ \hline
    \text{r103} & 1,213.62 & 15 & \text{1,208.70} & 14 & \text{0.41} \\ \hline
    \text{r104} & 976.61 & 11 & \text{971.50} & 11 & \text{0.53} \\ \hline
    \text{r105} & 1,360.78 & 15 & \text{1,355.30} & 15 & \text{0.40} \\ \hline
    \text{r106} & 1,239.37 & 13 & \text{1,234.60} & 13 & \text{0.39} \\ \hline
    \text{r107} & 1,073.60 & \text{12} & \text{1,064.60} & \text{11} & \text{0.85} \\ \hline
    \text{r108} & 944.44 & 10 & \text{932.10} & 10 & \text{1.32} \\ \hline
    \text{r109} & 1,152.38 & 13 & \text{1,146.90} & 13 & \text{0.48} \\ \hline
    \text{r110} & 1,078.59 & 12 & \text{1,068.00} & 12 & \text{0.99} \\ \hline
    \text{r111} & 1,053.50 & 12 & \text{1,048.70} & 12 & \text{0.46} \\ \hline
    \text{r112} & \text{955.68} & 10 & \text{948.60} & \text{10} & \text{0.75} \\ \hline
    avg &  &  &  &  & 0.61 \\ \hline
    \end{tabular}
    % \end{adjustbox}
  \end{table}


  \begin{table}[caption={Kết quả đo với tập Solomon R2}, label=exp:solomonR2]
    % \begin{adjustbox}{width=1\textwidth}
    \centering
    \begin{tabular}{lrrrll}
    \hline
    instance & \multicolumn{1}{l}{alns best} & \multicolumn{1}{l}{nv} & \multicolumn{1}{l}{bk cost} & bk nv & gap (\%) \\ \hline
    r201 & 1,152.96 & \textbf{7} & 1,143.20 & 8 & 0.32 \\ \hline
    r202 & 1,035.32 & \textbf{7} & 1,029.60 & 8 & 0.42 \\ \hline
    r203 & 880.90 & 6 & 870.80 & 6 & 0.41 \\ \hline
    r204 & 743.91 & \textbf{4} & 731.30 & 5 & 0.53 \\ \hline
    r205 & 958.81 & 5 & 949.80 & 5 & 0.40 \\ \hline
    r206 & 883.92 & 5 & 875.90 & 5 & 0.39 \\ \hline
    r207 & 806.31 & 5 & 794.00 & 4 & 0.85 \\ \hline
    r208 & 948.57 & 4 & 701.00 & 4 & 1.77 \\ \hline
    r209 & 717.53 & 5 & 854.80 & 5 & 0.48 \\ \hline
    r210 & 909.32 & \textbf{5} & 900.50 & 6 & 0.99 \\ \hline
    r211 & 1,053.50 & 5 & 746.70 & 4 & 0.46 \\ \hline
    avg &  &  &  &  & 1.38 \\ \hline
    \end{tabular}
    % \end{adjustbox}
  \end{table}

  Tập R1 và R2 có các yêu cầu được tạo hoàn toàn ngẫu nhiên thế nên cũng có nhiều nghiệm chấp nhận được và thuật toán cũng khó bị bẫy ở một nghiệm tối ưu cục bộ. Tuy nhiên, thuật toán cũng mất nhiều thời gian để tìm nghiệm tối ưu hơn do có quá nhiều nghiệm thỏa mãn các ràng buộc. Riêng với tập R2, ALNS đã tìm ra nghiệm với số xe ít hơn nghiệm tốt nhất đã biết mà tổng khoảng cách chỉ chênh lệch nhỏ. 

  \begin{table}[caption={Kết quả đo với tập Solomon RC1}, label=exp:solomonRC1]
    % \begin{adjustbox}{width=1\textwidth}
    \centering
    \begin{tabular}{lrrrll}
    \hline
    instance & \multicolumn{1}{l}{alns best} & \multicolumn{1}{l}{nv} & \multicolumn{1}{l}{bk cost} & bk nv & gap (\%) \\ \hline
    rc101 & 1,623.58 & 16 & 1,619.80 & 15 & 0.23 \\ \hline
    rc102 & 1,461.23 & 14 & 1,457.40 & 14 & 0.26 \\ \hline
    rc103 & 1,266.62 & 11 & 1,258.00 & 11 & 0.69 \\ \hline
    rc104 & 1,136.91 & 10 & 1,132.30 & 10 & 0.41 \\ \hline
    rc105 & 1,518.58 & 16 & 1,513.70 & 15 & 0.32 \\ \hline
    rc106 & 1,376.99 & 13 & 1,372.70 & 12 & 0.31 \\ \hline
    rc107 & 1,211.11 & 12 & 1,207.80 & 12 & 0.27 \\ \hline
    rc108 & 1,118.13 & 11 & 1,114.20 & 11 & 0.35 \\ \hline
    avg &  &  &  &  & 0.36 \\ \hline
    \end{tabular}
    % \end{adjustbox}
  \end{table}

  \begin{table}[caption={Kết quả đo với tập Solomon RC2}, label=exp:solomonRC2]
    % \begin{adjustbox}{width=1\textwidth}
    \centering
    \begin{tabular}{lrrrll}
    \hline
    instance & \multicolumn{1}{l}{alns best} & \multicolumn{1}{l}{nv} & \multicolumn{1}{l}{bk cost} & bk nv & gap (\%) \\ \hline
    rc201 & 1,274.61 & \textbf{8} & 1,261.80 & 9 & 1.02 \\ \hline
    rc202 & 1,099.54 & \textbf{6} & 1,092.30 & 8 & 0.66 \\ \hline
    rc203 & 931.16 & 5 & 923.70 & 5 & 0.81 \\ \hline
    rc204 & 788.66 & 4 & 783.50 & 4 & 0.66 \\ \hline
    rc205 & 1,157.66 & 7 & 1,154.00 & 7 & 0.32 \\ \hline
    rc206 & 1,060.50 & \textbf{6} & 1,051.10 & 7 & 0.89 \\ \hline
    rc207 & 966.08 & 6 & 962.90 & 6 & 0.33 \\ \hline
    rc208 & 785.73 & 4 & 776.10 & 4 & 1.24 \\ \hline
    avg &  &  &  &  & 0.74 \\ \hline
    \end{tabular}
    % \end{adjustbox}
  \end{table}

  Tập RC1 và RC2 chứa các yêu cầu ở các vị trí được phân cụm một cách tương đối (lai giữa tập R và tập C) được đánh giá là tập khó, do các vị trí của yêu cầu đủ ngẫu nhiên để bài toán có nhiều nghiệm chấp nhận được nhưng lại dễ bị bẫy ở nghiệm tối ưu cục bộ khi mà xe đã phụ vụ các yêu cầu ở trong một cụm chỉ cần ta bỏ đi một (vài) yêu cầu "quan trọng" trong cụm là đã nhận được một nghiệm tệ hơn trước khá nhiều. Mặc dù vậy ALNS vẫn tỏ ra hiệu quả khi tìm được nghiệm với chênh lệch trung bình $0.36\%$ và $0.74\%$ cho RC1 và RC2. Ngoài ra, ALNS cũng tìm được nghiệm với số xe ít hơn nghiệm tốt nhất đã biết trong một số cấu hình của tập RC2.
\section{Hiệu năng thuật toán}
Trong phần này chúng ta sẽ xem xét hiệu năng của ALNS. Để thấy rõ sự khác biệt, các đồ thị được trình bày dưới đây là kết quả cho cấu hình với số lượng yêu cầu rất lớn ($1000$ yêu cầu) và số lượng yêu cầu trung bình ($400$ yêu cầu). Ngoài ra, để tránh dài dòng, các đồ thị được thể hiện cho 3 cầu hình C1\_x\_1, R1\_x\_1, RC1\_x\_1 ($x = 10$ cho tập $1000$ yêu cầu, và $x=4$ cho tập $400$ yêu cầu). Hiệu năng của các thuật toán đối với các cấu hình khác là tương tự. Chúng ta sẽ so sánh hiệu năng của ALNS nguyên bản, B-ALNS và một framework rất nổi tiếng trong cộng đồng là \code{Google OR-Tools}. Hiệu năng được so sánh cho cả phiên bản đơn luồng và đa luồng của ALNS.

\subsection{Giá trị hàm mục tiêu}

Trước hết, giá trị hàm mục tiêu theo thời gian được xem xét. Các đồ thị dưới đây cho thấy giá trị hàm mục tiêu theo thời gian khi chạy thuật toán với thời gian chạy 60 giây và được phóng đại để nhìn rõ hơn sự khác biệt trong 10 giây đầu tiên. Đồ thị được chia làm 3 phần tương ứng với 3 cấu hình khác nhau.

\begin{figure}[H] % places figure environment here   
  \label{fig:perf_ct_c1_10}
  \begin{subfigure}{.5\textwidth}
    \centering
    \includegraphics[width=1\linewidth]{figures/cost_time_60s_C1_10_1.png}
    \caption{60s}
    \label{fig:perf_ct_c1_10_60s}
  \end{subfigure}%
  \begin{subfigure}{.5\textwidth}
    \centering
    \includegraphics[width=1\linewidth]{figures/cost_time_10s_C1_10_1.png}
    \caption{10s}
    \label{fig:perf_ct_c1_10_10s}
  \end{subfigure}
  \caption{Giá trị hàm mục tiêu theo thời gian, cấu hình C1\_10\_1}
\end{figure}

\begin{figure}[H] % places figure environment here   
  \label{fig:perf_ct_r1_10}
  \begin{subfigure}{.5\textwidth}
    \centering
    \includegraphics[width=1\linewidth]{figures/cost_time_60s_R1_10_1.png}
    \caption{60s}
    \label{fig:perf_ct_r1_10_60s}
  \end{subfigure}%
  \begin{subfigure}{.5\textwidth}
    \centering
    \includegraphics[width=1\linewidth]{figures/cost_time_10s_R1_10_1.png}
    \caption{10s}
    \label{fig:perf_ct_r1_10_10s}
  \end{subfigure}
  \caption{Giá trị hàm mục tiêu theo thời gian, cấu hình R1\_10\_1}
\end{figure}

\begin{figure}[H] % places figure environment here   
  \label{fig:perf_ct_rc1_10}
  \begin{subfigure}{.5\textwidth}
    \centering
    \includegraphics[width=1\linewidth]{figures/cost_time_60s_RC1_10_1.png}
    \caption{60s}
    \label{fig:perf_ct_rc1_10_60s}
  \end{subfigure}%
  \begin{subfigure}{.5\textwidth}
    \centering
    \includegraphics[width=1\linewidth]{figures/cost_time_10s_RC1_10_1.png}
    \caption{10s}
    \label{fig:perf_ct_rc1_10_10s}
  \end{subfigure}
  \caption{Giá trị hàm mục tiêu theo thời gian, cấu hình RC1\_10\_1}
\end{figure}

\begin{figure}[H] % places figure environment here   
  \label{fig:perf_ct_c1_4}
  \begin{subfigure}{.5\textwidth}
    \centering
    \includegraphics[width=1\linewidth]{figures/cost_time_10s_C1_4_1.png}
    \caption{10s}
    \label{fig:perf_ct_c1_4_10s}
  \end{subfigure}%
  \begin{subfigure}{.5\textwidth}
    \centering
    \includegraphics[width=1\linewidth]{figures/cost_time_1s_C1_4_1.png}
    \caption{1s}
    \label{fig:perf_ct_c1_4_1s}
  \end{subfigure}
  \caption{Giá trị hàm mục tiêu theo thời gian, cấu hình C1\_4\_1}
\end{figure}

\begin{figure}[H] % places figure environment here   
  \label{fig:perf_ct_r1}
  \begin{subfigure}{.5\textwidth}
    \centering
    \includegraphics[width=1\linewidth]{figures/cost_time_10s_R1_4_1.png}
    \caption{10s}
    \label{fig:perf_ct_r1_60s}
  \end{subfigure}%
  \begin{subfigure}{.5\textwidth}
    \centering
    \includegraphics[width=1\linewidth]{figures/cost_time_1s_R1_4_1.png}
    \caption{1s}
    \label{fig:perf_ct_r1_10s}
  \end{subfigure}
  \caption{Giá trị hàm mục tiêu theo thời gian, cấu hình R1\_4\_1}
\end{figure}

\begin{figure}[H] % places figure environment here   
  \label{fig:perf_ct_rc1_4}
  \begin{subfigure}{.5\textwidth}
    \centering
    \includegraphics[width=1\linewidth]{figures/cost_time_10s_RC1_4_1.png}
    \caption{10s}
    \label{fig:perf_ct_rc1_4_10s}
  \end{subfigure}%
  \begin{subfigure}{.5\textwidth}
    \centering
    \includegraphics[width=1\linewidth]{figures/cost_time_1s_RC1_4_1.png}
    \caption{1s}
    \label{fig:perf_ct_rc1_4_1s}
  \end{subfigure}
  \caption{Giá trị hàm mục tiêu theo thời gian, cấu hình RC1\_4\_1}
\end{figure}

Với thời gian chạy lâu, các thuật toán đều chững lại bởi khi đó các tuyến đường đã khá chật chội, việc "sửa chữa" là khó khăn hơn giai đoạn đầu rất nhiều. Nói cách khác, thuật toán bị bẫy trong nghiệm tối ưu cục bộ. Khi chạy thuật toán với thời gian lâu hơn (timeout 10 phút), tác giả nhận thấy hàm mục tiêu không giảm đáng kể nữa. Chính vì thế, thời gian chạy 60 giây được lựa chọn để vừa phù hợp với thực tế và tránh việc chạy quá lâu.

Ta thấy một xu hướng rõ ràng, trong giai đoạn đầu, B-ALNS tăng tốc hiệu năng của ALNS một cách đáng kể. Với tập dữ liệu $1000$ yêu cầu, thời gian chạy dưới 10 giây, B-ALNS đơn luồng cho hiệu năng gần như tương đương với ALNS với 4 luồng. Nghĩa là, B-ALNS tiết kiệm khoảng $75\%$ tài nguyên CPU để đạt kết quả tương đương với ALNS trong thời gian chạy dưới 10 giây. Chưa kể, memory cũng được tiết kiệm một cách đáng kể khi B-ALNS sử dụng 1 luồng thay vì 4 luồng. Khi so sánh với ALNS đơn luồng, B-ALNS đơn luồng cho hiệu năng vượt trội. Đối với tập dữ liệu có số lượng yêu cầu trung bình ($400$ yêu cầu), B-ALNS đơn luồng không có lợi thế so với ALNS đơn luồng, tuy nhiên hiệu năng đa luồng vẫn tốt hơn ALNS. Điều này cho thấy, B-ALNS có thể được sử dụng để giải quyết các bài toán lớn với tài nguyên hạn chế. Với cấu hình $400$ yêu cầu, B-ALNS đa luồng cho hiệu năng bỏ xa ALNS với thời gian chạy dưới $0.2$ giây và luôn tốt hơn trong thời gian chạy dưới $1$ giây. 

Khi so sánh với \code{Google OR-Tools}, ta thấy rằng, \code{Google OR-Tools} cho hiệu năng tốt trong thời gian chạy từ 2 đến 4 giây đầy tiên (tập $1000$ yêu cầu). Tuy nhiên với thời gian chạy dưới 1 giây, \code{Google OR-Tools} không thể cho ra kết quả. Với thời gian chạy lâu hơn, \code{Google OR-Tools} cho hiệu năng không tốt bằng ALNS hay B-ALNS. Với cấu hình $400$ yêu cầu, nhìn chung hiệu năng của B-ALNS cũng như ALNS bỏ xa \code{Google OR-Tools}.

Như vậy đối với các nghiệp vụ yêu cầu một kết quả tốt trong thời gian ngắn, B-ALNS là lựa chọn tốt nhất. Khi tiến hành đo đạc với thời gian chạy dài (timeout lớn hơn 1 phút), tác giả nhận thấy rằng, ALNS là tốt nhất trong các thuật toán được đề cập ở đây. Các kết quả được chỉ ra trong phần trước là giá trị hàm mục tiêu khi sử dụng ALNS nguyên bản.

\subsection{Số xe}

Chúng ta cũng nhận thấy một xu hướng tương tự như khi so sánh hiệu năng của các thuật toán khi sử dụng độ đo là giá trị hàm mục tiêu. B-ALNS cho hiệu năng vượt trội trong giai đoạn đầu. B-ALNS đơn luồng cho hiệu năng tương đương với ALNS với 4 luồng. Với cấu hình lớp R ($1000$ yêu cầu), B-ALNS tiết kiệm được khoảng $30\%$ số xe so với ALNS trong thời gian chạy từ 2 tới 4 giây! Tương tự, với cấu hình $400$ yêu cầu, B-ALNS giảm số xe nhanh hơn đáng kể so với ALNS trong thời gian chạy ngắn (dưới $0.2$ giây).  Việc tiết kiệm hàng trăm xe có ý nghĩa rất lớn trong thực tế. Thường thì chi phí cho một xe (thuê, hoặc mua, nhiên liệu, chi phí cho tài xế, ...) là rất lớn. Việc tiết kiệm số xe sẽ giúp giảm chi phí vận hành của doanh nghiệp một cách đáng kể. 

Với thời gian chạy lâu hơn, đương nhiên chúng ta rất khó để giảm được số xe nữa, vì hầu hết các tuyến đường đến lúc này đã chật chội hơn đáng kể so với giai đoạn đầu.

\begin{figure}[H] % places figure environment here   
  \label{fig:perf_ct_c1_10}
  \begin{subfigure}{.5\textwidth}
    \centering
    \includegraphics[width=1\linewidth]{figures/nv_time_60s_C1_10_1.png}
    \caption{60s}
    \label{fig:perf_ct_c1_10_60s}
  \end{subfigure}%
  \begin{subfigure}{.5\textwidth}
    \centering
    \includegraphics[width=1\linewidth]{figures/nv_time_10s_C1_10_1.png}
    \caption{10s}
    \label{fig:perf_ct_c1_10_10s}
  \end{subfigure}
  \caption{Số xe sử dụng theo thời gian, cấu hình C1\_10\_1}
\end{figure}

\begin{figure}[H] % places figure environment here   
  \label{fig:perf_ct_r1_10}
  \begin{subfigure}{.5\textwidth}
    \centering
    \includegraphics[width=1\linewidth]{figures/nv_time_60s_R1_10_1.png}
    \caption{60s}
    \label{fig:perf_ct_r1_10_60s}
  \end{subfigure}%
  \begin{subfigure}{.5\textwidth}
    \centering
    \includegraphics[width=1\linewidth]{figures/nv_time_10s_R1_10_1.png}
    \caption{10s}
    \label{fig:perf_ct_r1_10_10s}
  \end{subfigure}
  \caption{Số xe sử dụng theo thời gian, cấu hình R1\_10\_1}
\end{figure}

\begin{figure}[H] % places figure environment here   
  \label{fig:perf_ct_rc1}
  \begin{subfigure}{.5\textwidth}
    \centering
    \includegraphics[width=1\linewidth]{figures/nv_time_60s_RC1_10_1.png}
    \caption{60s}
    \label{fig:perf_ct_rc1_60s}
  \end{subfigure}%
  \begin{subfigure}{.5\textwidth}
    \centering
    \includegraphics[width=1\linewidth]{figures/nv_time_10s_RC1_10_1.png}
    \caption{10s}
    \label{fig:perf_ct_rc1_10s}
  \end{subfigure}
  \caption{Số xe sử dụng theo thời gian, cấu hình RC1\_10\_1}
\end{figure}

\begin{figure}[H] % places figure environment here   
  \label{fig:perf_ct_c1_4}
  \begin{subfigure}{.5\textwidth}
    \centering
    \includegraphics[width=1\linewidth]{figures/nv_time_10s_C1_4_1.png}
    \caption{10s}
    \label{fig:perf_ct_c1_4_10s}
  \end{subfigure}%
  \begin{subfigure}{.5\textwidth}
    \centering
    \includegraphics[width=1\linewidth]{figures/nv_time_1s_C1_4_1.png}
    \caption{1s}
    \label{fig:perf_ct_c1_4_1s}
  \end{subfigure}
  \caption{Số xe sử dụng theo thời gian, cấu hình C1\_4\_1}
\end{figure}

\begin{figure}[H] % places figure environment here   
  \label{fig:perf_ct_r1_4}
  \begin{subfigure}{.5\textwidth}
    \centering
    \includegraphics[width=1\linewidth]{figures/nv_time_10s_R1_4_1.png}
    \caption{10s}
    \label{fig:perf_ct_r1_4_10s}
  \end{subfigure}%
  \begin{subfigure}{.5\textwidth}
    \centering
    \includegraphics[width=1\linewidth]{figures/nv_time_1s_R1_4_1.png}
    \caption{1s}
    \label{fig:perf_ct_r1_4_1s}
  \end{subfigure}
  \caption{Số xe sử dụng theo thời gian, cấu hình R1\_4\_1}
\end{figure}

\begin{figure}[H] % places figure environment here   
  \label{fig:perf_ct_rc1}
  \begin{subfigure}{.5\textwidth}
    \centering
    \includegraphics[width=1\linewidth]{figures/nv_time_10s_RC1_4_1.png}
    \caption{10s}
    \label{fig:perf_ct_rc1_4_10s}
  \end{subfigure}%
  \begin{subfigure}{.5\textwidth}
    \centering
    \includegraphics[width=1\linewidth]{figures/nv_time_1s_RC1_4_1.png}
    \caption{1s}
    \label{fig:perf_ct_rc1_4_1s}
  \end{subfigure}
  \caption{Số xe sử dụng theo thời gian, cấu hình RC1\_4\_1}
\end{figure}