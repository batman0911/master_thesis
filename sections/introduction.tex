\chapter{Mở đầu}
Từ xưa tới nay, giao vận luôn là một trong những ngành đóng vai trò quan trọng trong nền kinh tế. Nó đóng vai trò là một cầu nối giữa các đơn vị sản xuất và người tiêu dùng. Nó cũng là một trong những ngành có ảnh hưởng lớn đến sự phát triển của một quốc gia. 

Từ những thập niên 90 của thế kỉ trước, thời kì bùng nổ của internet đã thúc đẩy một hình thức bán hàng hoàn toàn mới, đó là bán hàng trực tuyến. Hàng loạt các sàn thương mại điện tử lớn ra đời, có thể kể đến như Amazon (Mỹ - 1994), Alibaba (Trung Quốc - 1999), Rakuten (Nhật Bản - 1997). Ngày nay các sàn thương mại điện tử này trở thành những công ty hàng đầu thế giới không chỉ ở lĩnh vực bán hàng mà là cả công nghệ. Việc phát triển vũ bão của các sàn thương mại điện tử dẫn đến số lượng hàng hóa được tiêu thụ trên toàn cầu tăng lên một cách đáng kinh ngạc so với bán hàng truyền thống. Logistic và quản lý chuỗi cung ứng là một trong những xương sống của thương mại điện tử cùng với \textit{nền tảng công nghệ}, \textit{nền tảng thanh toán} hay \textit{chăm sóc khách hàng}... Để quản lý, giao vận số lượng đơn hàng lớn như vậy tới tay khách hàng, cách thức làm việc trong ngành logistic cũng phải thay đổi rất nhiều, áp dụng những công nghệ hiện đại hơn cách làm truyền thống.
