\chapter{Mô hình toán học}
Chương này trình bày biểu diễn toán học cho bài toán VRPTW. Trong luận văn này, tác giả tập trung giải quyết VRPTW, từ đó ta cũng có thể giản ước về CVRP cũng như tổng quát với VRPPD (\textit{VRP with pickup and delevery}) hoặc PDPTW (\textit{pickup and delivery with time window}). Như đã trình bày ở chương trước VRPTW là một mở rộng của CVRP với ràng buộc khung thời gian. Trong đó mỗi khách hàng $i$ được ràng buộc bởi một khung thời gian $[a_i,b_i]$. Xe không được đến $i$ tại thời điểm $t_i > b_i$, ngoài ra nếu đến sớm hơn thởi điểm $a_i$ hay $t_i < a_i$ thì xe cần phải chờ tới thời điểm $a_i$ để phục vụ khách hàng. Thời gian phục vụ của khách hàng $i$ là $s_i$.

VRPTW là bài toán NP-khó, việc tìm lời giải hay nghiệm tối ưu (chính xác) gần như là bất khả thi. Để dễ hình dung, xét bài toán VRP, với số lượng khách hàng $n=100$, và chỉ một xe, số lượng lời giải là $n! \approx 10^{158}$. Nếu ta có số CPU ước tính bằng toàn bộ số nguyên tử trong vũ trụ $n_{CPU} \approx 10^{80}$, thời gian nhỏ nhất là thời gian Plank $t_p \approx 5.39 \times 10^{-44}$. Để kiểm tra toàn bộ lời giải có phải nghiệm tối ưu ta cần thời gian $T \approx 10^{158} \times 5.39 \times 10^{-44} / 10^{80} \approx 5.39 \times 10^{34}$. Để so sánh, tuổi của vũ trụ được ước tính khoảng $4.33 \times 10^{17}$. Nghĩa là ta sẽ mất thời gian lớn gấp cỡ \textit{một trăm triệu tỉ} lần tuổi của cả vũ trụ! \footnote[1]{Slides của Thibaut Vidal (SOICT, Nha Trang 2017)}