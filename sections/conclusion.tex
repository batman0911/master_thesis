\chapter{Kết luận}
\label{chap:conclusion}

Trong luận văn này, tác giả đã nghiên cứu mô hình toán học, các thuật toán giải lớp các bài toán định tuyến xe, đi sâu vào phương pháp tìm kiếm lân cận rộng. Thuật toán \textit{Tìm kiếm lân cận rộng thích ứng - ALNS} được sử dụng để giải bài toán \textit{Định tuyến xe với ràng buộc tải trọng và khung thời gian - VRPTW}. Tác giả đã đề xuất một hiệu chỉnh giúp tăng hiệu năng của ALNS một cách đáng kể được chỉ ra trong kết quả thực nghiệm. Thuật toán được đặt tên là B-ALNS (\textit{Boosted - Adaptive Large Neighborhood Search - Tìm kiếm lân cận rộng thích ứng tăng tốc}). B-ALNS được đánh giá trên các tập dữ liệu thực tế với số lượng yêu cầu (khách hàng) từ trung bình tới rất lớn cho hiệu năng vượt xa ALNS gốc cũng như bộ thư viện phổ biến \code{Google OR-Tools}. Điều này gợi ý sử dụng B-ALNS cho các bài toán thực tế yêu cầu chất lượng nghiệm tốt trong thời gian ngắn. Ngoài ra B-ALNS cũng tiết kiệm tài nguyên tính toán (CPU, bộ nhớ) đáng kể so với ALNS gốc.

Tác giả cũng đã đưa ra gợi ý áp dụng B-ALNS cho tình huống thực tế khi ta cần một phương án định tuyến tốt trong thời gian ngắn. Về lâu dài, nếu mục đích cao nhất là giảm chi phí mà không cần quan tâm tới thời gian chạy, ta nên sử dụng ALNS gốc để đạt được chất lượng nghiệm tốt hơn. Một gợi ý khác là ta có thể sử dụng B-ALNS ở giai đoạn đầu giúp giảm nhanh hàm mục tiêu, sau đó áp dụng ALNS cho giai đoạn sau để tìm nghiệm tốt hơn. Thời điểm chuyển đổi từ B-ALNS sang ALNS có thể là khi hàm mục tiêu giảm không đủ nhiều theo thời gian hay số vòng lặp nữa (theo một ngưỡng nào đó). ALNS cũng như B-ALNS thể hiện tốt về cả chất lượng nghiệm, số xe sử dụng cũng như hiệu năng và độ ổn định khi giải VRPTW với các cấu hình có số lượng yêu cầu không quá lớn (từ $600$ yêu cầu trở xuống). Với các cấu hình mà số lượng yêu cầu là rất lớn (trên $600$ yêu cầu), B-ALNS và ALNS cho chất lượng nghiệm chấp nhận được trong khoảng thời gian chạy ngắn (timeout dưới một phút).

Mở rộng nghiên cứu, tác giả sẽ tiếp tục cải tiến hiệu chỉnh B-ALNS để thu được nghiệm chất lượng hơn cũng như hiệu năng cao hơn nữa. Các tham số khác ngoài số vòng lặp có thể được đưa vào nhiễu hiệu chỉnh trọng số lựa chọn thuật toán. Thời điểm chuyển đổi B-ALNS và ALNS như đã trình bày ở trên cũng sẽ được nghiên cứu tiếp để đưa ra một chiến thuật thích ứng hợp lý để thuật toán vừa chạy nhanh mà vẫn thu được nghiệm tốt. Thêm vào đó, tác giả tiếp tục nghiên cứu cơ chế thích ứng để "bỏ đi" và "thêm lại" yêu cầu với số lượng hợp lý theo trạng thái của hệ nhằm cải thiện chất lượng nghiệm cũng như tăng hiệu năng.

