% 	This template is  MIT licensed.

% 	Basic file to demonstrate the usage of this LaTeX template.
% 	You can build your own paper/thesis on top of this file.
% 	Simply adjust the document class and all metadata and start working.
%
\documentclass[
	language=english, % set to english or german
	type=master, % set to bachelor, master or seminar
]{isthesis}

\usepackage[utf8]{vntex}

% custom package
\usepackage{eurosym}
\usepackage{makecell}
\usepackage{hyperref}
\usepackage[utf8]{inputenc}
\usepackage{tabto} 
\usepackage{longtable}
\usepackage{algorithm} 
\usepackage{algpseudocode}
\usepackage{biblatex} %Imports biblatex package
% \usepackage{multirow}
% \usepackage{floatrow}

% Graphics rendering using TikZ
% See: https://en.wikibooks.org/wiki/LaTeX/PGF/TikZ
\usepackage{tikz}
\usepackage{xcolor}
\definecolor{light-gray}{gray}{0.95}
\newcommand{\code}[1]{\colorbox{light-gray}{\texttt{#1}}}
% Include required TikZ libraries here, some exemplary libraries are pre-included
\usetikzlibrary{calc}
\usetikzlibrary{matrix}
\usetikzlibrary{positioning}
\usetikzlibrary{shapes.geometric}

\usepackage{amsthm}
% \usepackage{ntheorem}
\newtheoremstyle{theoremst}% name of the style to be used
  {\topsep}% measure of space to leave above the theorem. E.g.: 3pt
  {\topsep}% measure of space to leave below the theorem. E.g.: 3pt
  {\normalfont}% name of font to use in the body of the theorem
  {0pt}% measure of space to indent
  {\bfseries}% name of head font
  {:}% punctuation between head and body
  { }% space after theorem head; " " = normal interword space
  {\thmname{#1}\thmnumber{ #2}\textnormal{\thmnote{ (#3)}}}

\newtheoremstyle{examplest}% name of the style to be used
  {\topsep}% measure of space to leave above the theorem. E.g.: 3pt
  {\topsep}% measure of space to leave below the theorem. E.g.: 3pt
  {\normalfont}% name of font to use in the body of the theorem
  {0pt}% measure of space to indent
  {\bfseries}% name of head font
  {\\}% punctuation between head and body
  { }% space after theorem head; " " = normal interword space
  {\thmname{#1}\thmnumber{ #2}\textnormal{\thmnote{ (#3)}}}


% \theorembodyfont{\normalfont}
% \theoremseparator{:}
\theoremstyle{theoremst}
\newtheorem{definition}{Định nghĩa}[section]
\newtheorem{theorem}{Định lý}[section]
\newtheorem{proposition}{Khẳng định}[section]

% \theoremstyle{break}
% \theorembodyfont{\normalfont}
\theoremstyle{examplest}
\newtheorem{remark}{Nhận xét}[section]
\newtheorem{example}{Ví dụ}[section]

% \theoremstyle{remark}
% \newtheorem*{remark}{Remark}

%Add your library here
\addbibresource{refs.bib}

% Import acronyms
% \newacronym[longplural={<long plural>}, shortplural={<short plural>}]{<label>}{<short>}{<long>}
% 	label = is the unique identifier and sort key for the acronym, can be the same as <short>
%	short = is the abbreviation or acronym
%	short plural (optional) = is the plural of the abbreviation or acronym
%	long = is the long form of the acronym, this will appear in the list of abbreviations
%	long plural (optional) = is the long plural form of the abbreviation or acronym

\newacronym[shortplural={KMUen}, longplural={Kleine und Mittlere Unternehmen}]{kmu}{KMU}{Kleines und Mittleres Unternehmen}
\newacronym{CD}{CD}{Corporate Design}
\newacronym{SQL}{SQL}{Structured Query Language}
\newacronym{FAU}{FAU}{Khoa Toán cơ tin}
\newacronym{BPM}{BPM}{Business Process Management}
\newacronym{npm}{NPM}{Node Package Manager}
\newacronym{diss}{DISS}{Digital Industrial Service System}

% Import symbols
% Syntax: <Symbol> <Label> <Name>
% The symbols are sorted by their labels
\addsymboltolist{$\Pi$}{Pi}{Projection}
\addsymboltolist{$\Join$}{Join}{Natural Join}
\addsymboltolist{$\sigma$}{Selection}{Selection}


% Import custom commands
% If you want to define custom commands, please do so here

% Import custom code block
% define listing code
\definecolor{codegreen}{rgb}{0,0.6,0}
\definecolor{codegray}{rgb}{0.5,0.5,0.5}
\definecolor{codepurple}{rgb}{0.58,0,0.82}
\definecolor{backcolour}{rgb}{0.95,0.95,0.92}

\lstdefinestyle{code}{
    backgroundcolor=\color{backcolour},   
    commentstyle=\color{codegreen},
    keywordstyle=\color{magenta},
    numberstyle=\tiny\color{codegray},
    stringstyle=\color{codepurple},
    basicstyle=\ttfamily\footnotesize,
    breakatwhitespace=false,         
    breaklines=true,                 
    captionpos=b,                    
    keepspaces=true,                 
    numbers=left,
    firstnumber=1,
    stepnumber=1,                    
    numbersep=5pt,                  
    showspaces=false,                
    showstringspaces=false,
    showtabs=false,                  
    tabsize=2,
    framesep=10pt,
    xleftmargin=10pt,
    xrightmargin=10pt,
    framexleftmargin=16pt,
    framextopmargin=2pt,
    framexbottommargin=2pt, 
    frame=tb, framerule=0pt,
}

\lstdefinestyle{algo}{
    backgroundcolor=\color{backcolour},   
    commentstyle=\color{codegreen},
    keywordstyle=\color{magenta},
    numberstyle=\tiny\color{codegray},
    stringstyle=\color{codepurple},
    basicstyle=\ttfamily\footnotesize\small\linespread{0.8},
    breakatwhitespace=false,         
    breaklines=true,                 
    captionpos=b,                    
    keepspaces=true,                 
    numbers=none,
    firstnumber=1,
    stepnumber=1,                    
    numbersep=5pt,                  
    showspaces=false,                
    showstringspaces=false,
    showtabs=false,                  
    tabsize=2,
    framesep=10pt,
    xleftmargin=10pt,
    xrightmargin=10pt,
    framexleftmargin=16pt,
    framextopmargin=2pt,
    framexbottommargin=2pt, 
    frame=tb, framerule=0pt,
    mathescape=true
}

\lstset{style=code}

% Document meta information
\isthesis{
    title={LUẬN VĂN THẠC SĨ},
    sub-title={Phương pháp tìm kiếm lân cận rộng thích ứng \\  
    cho bài toán định tuyến xe},
    author-name={Nguyễn Mạnh Linh}, 
    % Separate multiple authors with commas
    % author-email={linhnguyen.code@gmail.com},
    % author-matriculation={MATRICULATION NUMBER},
    % author-phone={+49 XXXXXXXXX}, % Use international numbers format
    % author-address={STREET},
    % author-zip={ZIP},
    % author-city={CITY},
    principal-supervisor={TS. Hoàng Nam Dũng}, % This must be a professor
    % associate-supervisor={SUPERVISOR}, % This is your main supervisor, i.e., a post doc or doctoral student
    tutor-supervisor={}, % If required, define an additional supervisor resp. tutor here
    group-institute={Trường Đại học Khoa học Tự nhiên},
    % group={Image Data Exploration and Analysis (IDEA) Lab},
    % studies={M.Sc. International Information Systems}, %your field of studies, i.e. Wirtschaftsinformatik or International Information Systems
    %
    %associate-group={}, % When the thesis is done in cooperation with another chair, add it here
    %associate-group-institute={}, % add cooperating institute or university here
    seminar={SEMINAR}, % The title of your seminar
    submission-date={2023-11-01}, % The date you handed in your document: Format yyyy-mm-dd
    primary-logo={assets/hus.png}, % Uses the FAU logo by default
    %primary-logo-height={}, % Uses 16mm as default height
    %secondary-logo={}, % Logo of the secondary institution (cooperating chair/university), USES Faculty logo by default
    %secondary-logo-height={} % Uses 16mm as default height
}


\begin{document}
    % Title page
    \newcounter{savepage}
    \maketitle

	% Quote
    % You can put an optional quote page in front of your content
    %   \quotepage[author={Arthur C. Clarke}]{
    %   	        Any sufficiently advanced technology is indistinguishable from magic.
    %   }
    
    % Table of contents
    \tableofcontents

    % \begin{abstract}
   Trong luận văn này, tác giả nghiên cứu mô hình toán học cho lớp các bài toán định tuyến xe (\textit{Vehicle Routing Problem} - VRP). Các thuật toán liên quan được giới thiệu và phân loại xuyên suốt lịch sử hơn $60$ năm phát triển của VRP. Thuật toán \textit{Tìm kiếm lân cận rộng thích ứng} (\textit{Adaptive Large Neighborhood Search}) được sử dụng để giải bài toán \textit{Định tuyến xe với ràng buộc khung thời gian} (\textit{Vehicle Routing Problem with Time Window - VRPTW}). Hai thuật toán hủy mới được tác giả phát triển giúp giảm nhanh số xe sử dụng và tăng hiệu năng khi xóa yêu cầu. Thêm vào đó, tác giả đề xuất một hiệu chỉnh cho ALNS được gọi là B-ALNS (\textit{Boosted - Adaptive Large Neighborhood Search}). Các thuật toán được đánh giá trên các tập dữ liệu với số lượng yêu cầu từ nhỏ ($100$ yêu cầu) tới rất lớn ($1000$ yêu cầu). B-ALNS được chỉ ra có hiệu năng vượt trội so với ALNS (lên tới hơn $30\%$) và tiết kiệm tới $75\%$ tài nguyên tính toán cho một số cấu hình trong khoảng thời gian nhất định. Ngoài ra, tác giả đưa ra gợi ý áp dụng ALNS, B-ALNS cho các tình huống thực tế với mục đích khác nhau.
\end{abstract} 

    \chapter{Mở đầu}
Từ xưa tới nay, giao vận luôn là một trong những ngành đóng vai trò quan trọng trong nền kinh tế. Nó đóng vai trò là một cầu nối giữa các đơn vị sản xuất và người tiêu dùng. Nó cũng là một trong những ngành có ảnh hưởng lớn đến sự phát triển của một quốc gia. 

Từ những thập niên 90 của thế kỉ trước, thời kì bùng nổ của internet đã thúc đẩy một hình thức bán hàng hoàn toàn mới, đó là bán hàng trực tuyến. Hàng loạt các sàn thương mại điện tử lớn ra đời, có thể kể đến như Amazon (Mỹ - 1994), Alibaba (Trung Quốc - 1999), Rakuten (Nhật Bản - 1997). Ngày nay các sàn thương mại điện tử này trở thành những công ty hàng đầu thế giới không chỉ ở lĩnh vực bán hàng mà là cả công nghệ. Việc phát triển vũ bão của các sàn thương mại điện tử dẫn đến số lượng hàng hóa được tiêu thụ trên toàn cầu tăng lên một cách đáng kinh ngạc so với bán hàng truyền thống. Logistic và quản lý chuỗi cung ứng là một trong những xương sống của thương mại điện tử cùng với \textit{nền tảng công nghệ}, \textit{nền tảng thanh toán} hay \textit{chăm sóc khách hàng}... Để quản lý, giao vận số lượng đơn hàng lớn như vậy tới tay khách hàng, cách thức làm việc trong ngành logistic cũng phải thay đổi rất nhiều, áp dụng những công nghệ hiện đại hơn cách làm truyền thống.


    % \begin{abstract}
	%     % Add your abstract here:
        
	% 	% \lipsum[1]
	% \end{abstract}

    % List of figures (if you have figures)
    % \listoffigures

    % List of tables (if you have tables)
    % \listoftables
    
    % List of listings (if you have listings)
	% \lstlistoflistings

    % List of abbreviations (if you use acronyms)
    %\listofabbreviations

    % List of symbols (if you use symbols)
    %\listofsymbols
	
	% Abstract
	%
	% Comment out this part, if you don't require an abstract

	
	% storing the last pagenumber
    \setcounter{savepage}{\value{page}}
    
    
    % Content
    \begin{content}
        % Add your content files:
        \chapter{Mô hình toán học}
Phần này, chúng ta sẽ cùng xây dựng mô hình toán học của bài toán định tuyến xe với ràng buộc thời gian. \cite{griersonRH}
    \end{content}
    
    \pagenumbering{Roman}
    \setcounter{page}{\numexpr\value{savepage}}

    % References
    \references{}

    
    % Appendix
    %  \begin{appendix}
    %     % In the appendices, use \section{} instead of \chapter{}
    %      \section{Some Appendix Section}
\label{sec:appendix01}
Appendices provide only two structural levels, viz., \texttt{\textbackslash section}, and \texttt{\textbackslash subsection}.

The numbering of figures, listings, tables, and footnotes is not reset. Thus, it continues as usual in the appendix.

\subsection{Some Appendix Subsection}

\lipsum[10]
    %  \end{appendix}




    % Declaration of authorship
    % \authorshipstatement[pagenumbering=false]
    % \authorshipstatement[pagenumbering=true]
    % \authorshipstatement[pagenumbering=only]
    
    % Consent form for use of plagiarism detection software
    % Not yet required
    % \consentform[pagenumbering=false]
    % \consentform[pagenumbering=true]
    % \consentform[pagenumbering=only]
    
    % Bonus: Wordcount
    % cd %FOLDER WHERE THE .tex FILES ARE IN %
    % clear
    % texcount -total -q -col -sum *.tex
    
\end{document}
